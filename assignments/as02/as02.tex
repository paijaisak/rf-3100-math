\documentclass[12pt, a4paper]{article}
\usepackage[utf8]{inputenc}
\usepackage[leqno]{amsmath}
\usepackage{amsfonts}
\usepackage[a4paper, total={6in, 8in}, margin=0.8in, tmargin=0in]{geometry}
%\usepackage{graphicx}
%\graphicspath{{./images}}

\title{Assignment 02}
\author{Embla Mathea}
\date{\today}

\begin{document}
\maketitle

\begin{center}
	\section*{Exercises}
\end{center}

\subsection*{Exercise 01}
Gitt et plan som går gjennom punktet
$A=\left(-3,\;1,\;\frac{1}{2}\right)$.
Normalvektoren til planet er $\left[2,\;5,\;-1\right]$.\\\\		
	(a)\quad Sett opp den implisitte formelen til planet.
			\begin{gather}
				\tag*{}
					ax+by+cz\;=\;d\\
				\tag*{}
					2x+5y-z\;=\;d\\
				\tag*{}
					2(-3)+5(1)-\frac{1}{2}\;=\;d\\
				\tag*{}
					-6+5-\frac{1}{2}\;=\;d\;=\;-\frac{3}{2}
			\end{gather}
			\begin{equation}
				\tag{Implisitt formel}
					2x+5y-z\;=\;-1.5\\
			\end{equation}
	(b)\quad Hva er avstanden mellom planet og origo?
			\begin{equation}
				\tag*{}
					\frac{d}{\sqrt{a^2+b^2+c^2}}\;=\;
					-\frac{1.5}{\sqrt{2^2+5^2+(-1)^2}}\;=\;
					-\frac{1.5}{\sqrt{30}}\;\approx\;0.274					
			\end{equation}
	(c)\quad Gjør om normalvektoren til en enhetsvektor
	med samme retning, og ta indreproduktet med
	$\vec{A}$. Hva observerer du?
			\begin{gather}
				\tag*{}
					\hat{n}\;=\;\left[
						\frac{2}{\left|n\right|},\;
						\frac{5}{\left|n\right|},\;
						\frac{-1}{\left|n\right|}
					\right]\\
				\tag*{}
						\left|n\right|\;=\;\sqrt{2^2+5^2+(-1)^2}
						\;=\;\sqrt{30}\\
				\tag*{}
					\hat{n}\;=\;\left[
						\frac{2}{\sqrt{30}},\;
						\frac{5}{\sqrt{30}},\;
						\frac{-1}{\sqrt{30}}
					\right]
				\end{gather}
			\begin{equation}
				\tag*{}
					\hat{n}\cdot\vec{a}\;=\;
					\left[
						\frac{2}{\sqrt{30}},\;
						\frac{5}{\sqrt{30}},\;
						\frac{-1}{\sqrt{30}}
					\right]\cdot
					\left[
						-3,\;1,\;\frac{1}{2}
					\right]\;=\;
					\left[
						\frac{-6}{\sqrt{30}}+
						\frac{5}{\sqrt{30}}+
						\frac{-0.5}{\sqrt{30}}
					\right]\;=\;-\frac{1.5}{\sqrt{30}}
			\end{equation}
			
\newpage\quad			
\subsection*{Exercise 03}
Gitt et plan beskrevet med:
	\begin{equation}
		\tag*{}
		2x-y+3z=2
	\end{equation}
Gitt et punkt $A=(3,\;3,\;-1)$ som ligger utenfor planet.\\\\
	(a)\quad Hva er avstanden mellom planet og punktet
			\begin{equation}
				\tag*{}
					|\vec{a}|\;=\;\frac{\vec{A}\cdot\vec{n}-d}
					{\left|\vec{n}\right|^2}
			\end{equation}
			\begin{equation}
				\tag*{}
					\vec{a}\;=\;\frac{2(3)-3+3(-1)-2}
					{\sqrt{2^2+(-1)^2+3^2}^2}\;=\;
					\frac{6-3-3-2}{14}\;=\;
					-\frac{2}{14}\;=\;-\frac{1}{7}
			\end{equation}
	(b)\quad Finn koordinatene til punktet på planet som
	ligger nærmest $A$.
			\begin{equation}
				\tag*{}
					\vec{P}\;=\;\vec{A}-|\vec{a}|\cdot\vec{n}
			\end{equation}
			\begin{gather}
				\tag*{}
					\vec{P}\;=\;\vec{A}-\frac{1}{7}(2,\;-1,\;3)
					\;=\;|3,\;3,\;-1|-\left|
						\frac{2}{7},\;
						\frac{-1}{7},\;
						\frac{3}{7}
					\right|\\
				\tag*{}
					\;=\;\left|
						\frac{21-2}{7},\;
						\frac{21-(-1)}{7},\;
						\frac{-7-3}{7}
					\right|\;=\;\left|
						\frac{19}{7},\;
						\frac{22}{7},\;
						-\frac{10}{7}
					\right|
			\end{gather}
			\begin{equation}
					\tag*{}
						P\;=\;\left(
							\frac{19}{7},\;
							\frac{22}{7},\;
							-\frac{10}{7}
						\right)
			\end{equation}
\newpage\quad		
\subsection*{Exercise 04}
Gitt fem punkter i rommet:
	\begin{gather}
		\tag*{}
			A\;=\;(1,\;2,\;0)\\
		\tag*{}
			B\;=\;\left(-2,\;-\frac{1}{3},\;2\right)\\
		\tag*{}
			C\;=\;(0,\;2,\;3)\\
		\tag*{}
			D\;=\;\left(3,\;\frac{1}{2},\;0\right)\\
		\tag*{}
			E\;=\;(1,\;1,\;2)
	\end{gather}
	(a)\quad Bruk formlene fra s. 315 i læreboka (eller
	algoritmen fra s. 315-316) til å finne normalvektoren til
	planet som passer best til disse punktene.
			\begin{gather}
				\tag*{}
					n_x\;=\;(z_1+z_2)(y_1-y_2)\;+\;
						(z_2+z_3)(y_2-y_3)\;+\;
						(z_3+z_4)(y_3-y_4)\;+\;
						(z_4+z_5)(y_4-y_5)\\
				\tag*{}
					n_y\;=\;(x_1+x_2)(z_1-z_2)\;+\;
						(x_2+x_3)(z_2-z_3)\;+\;
						(x_3+x_4)(z_3-z_4)\;+\;
						(x_4+x_5)(z_4-z_5)\\
				\tag*{}
					n_z\;=\;(y_1+y_2)(x_1-x_2)\;+\;
						(y_2+y_3)(x_2-x_3)\;+\;
						(y_3+y_4)(x_3-x_4)\;+\;
						(y_4+y_5)(x_4-x_5)
			\end{gather}
			\begin{gather}
				\tag*{}
					n_x\;=\;
						(0+2)\left(2-\left(-\frac{1}{3}
							\right)\right)\;+\;
						(2+3)\left(-\frac{1}{3}-2\right)\;+\;
						(3+0)\left(2-\frac{1}{2}\right)\;+\;
						(0+2)\left(\frac{1}{2}-1\right)\\
				\tag*{}
					n_y\;=\;
						(1+(-2))(0-2)\;+\;
						(-2+0)(2-3)\;+\;
						(0+3)(3-0)\;+\;
						(3+1)(0-2)\\
				\tag*{}
					n_z\;=\;
						\left(2+\left(-\frac{1}{3}\right)\right)
							(1-(-2))\;+\;
						\left(-\frac{1}{3}+2\right)(-2-0)\;+\;
						\left(2+\frac{1}{2}\right)(0-3)\;+\;
						\left(\frac{1}{2}+1\right)(3-1)
			\end{gather}
			\begin{gather}
				\tag*{}
					n_x\;=\;2\left(\frac{6}{3}+\frac{1}{3}
					\right)+5\left(-\frac{1}{3}-\frac{6}{3}
					\right)+3\left(\frac{4}{2}-\frac{1}{2}
					\right)+2\left(\frac{1}{2}-\frac{2}{2}
					\right)\\
				\tag*{}
					n_x\;=\;2\left(\frac{7}{3}\right)+
					5\left(-\frac{7}{3}\right)+
					3\left(\frac{3}{2}\right)+
					2\left(-\frac{1}{2}\right)
					\;=\;\frac{14}{3}-\frac{35}{3}+
					\frac{9}{2}-1\\
				\tag*{}
					\;=\;-7+\frac{9}{2}-1
					\;=\;-\frac{-14+9-2}{2}\;=\;-\frac{7}{2}
					\;=\;3.5
			\end{gather}
			\begin{equation}
				\tag*{}
					n_y\;=\;-1(-2)-2(-1)+3(3)+4(-2)\;=\;
					2+2+9-8\;=\;5
			\end{equation}
			\begin{gather}
				\tag*{}
					n_z\;=\;\frac{5}{3}(3)+\frac{5}{3}(-2)
					+\frac{5}{2}(-3)+\frac{3}{2}(2)\\
				\tag*{}
					\;=\;\frac{15}{3}-\frac{10}{3}
					-\frac{15}{2}+\frac{6}{2}\;=\;
					\frac{5}{3}-\frac{9}{2}\;=\;
					\frac{10-27}{6}\;=\;-\frac{17}{6}
			\end{gather}
			\begin{gather}
				\tag*{}
					\hat{n}\;=\;\left[
						\frac{3.5}{|n|},\;
						\frac{5}{|n|},\;
						\frac{\left(-\frac{17}{6}\right)}{|n|}
					\right]\\
				\tag*{}
					|n|\;=\;\sqrt{
						3.5^2+5^2+\left(-\frac{17}{6}\right)^2
					}\;=\;\sqrt{12.25+25+\frac{289}{36}}
					\;=\;\frac{\sqrt{1630}}{6}
					\;\approx\;6.728\\
				\tag*{}
					\hat{n}\;=\;\left[
							\frac{3.5}{6.728},\;
							\frac{5}{6.728},\;
							\frac{\left(-\frac{17}{6}\right)}
							{6.728}					
					\right]\;=\;\left[
						0.52,\;0.74,\;-0.42
					\right]
			\end{gather}
\newpage
	\begin{equation}
		\tag*{}
	\end{equation}
	(b)\quad Bruk formelene fra s. 316 til å finne $d$.
			\begin{gather}
				\tag*{}
					d\;=\;\frac{1}{n}\sum_{i=1}^{n}\left(
						p_i \cdot n
					\right)\;=\;\frac{1}{n}\left(
						\sum_{i=1}^{n}p_i
					\right)\cdot n\\
				\tag*{}
					\cdots
			\end{gather}
\subsection*{Exercise 05}
Gitt en trekant i 3D-rommet med hjørnene
$A$, $B$ og $C$. Vi får vite at $A$ og $B$ er henholdsvis $(2, 0, 1)$ og $(3, 3, 4)$. Vi får også
vite at vinkelen ved hjørnet $C$ er $50^\circ$. Til slutt får vi vite at lengden fra $B$ til $C$ er 
$4,5$\\\\
	(a)\quad Regn ut lengden $AB$.
			\begin{gather}
				\tag*{}
					\vec{AB}\;=\;[
						3-2,\;3-0,\;4-1
					]\;=\;[1,\;3,\;3]\\
				\tag*{}
					\left|\vec{AB}\right|\;=\;
					\sqrt{1^2+3^2+3^2}\;=\;\sqrt{19}
			\end{gather}
	(b)\quad Bruk sinussetningen for å beregne vinkelen
	ved hjørnet $A$.
			\begin{equation}
				\tag*{}
					\angle A\;=\;
			\end{equation}
	(d)\quad Bruk cosinussetningen for å beregne lengden $AC$.
			\begin{equation}
				\tag*{}
			\end{equation}

\newpage\quad
\subsection*{Exercise 06}
Gitt en trekant i 2D-rommet med hjørnene i A = (2, 5),
	B = (-1, -2) og C = (3, -0,5).\\\\
	(a)\quad Et punkt D har barisentriske koordinater $
		\left(\frac{1}{4}:\frac{2}{5}:\frac{1}{3}\right)
	$ i forhold til trekanten $ABC$. Finn de kartesiske 
	koordinatene til punktet.
			\begin{equation}
				\tag*{}
			\end{equation}
	(c)\quad Regn ut de kartesiske koordinatene til trekantens
	tyngdepunkt.
			\begin{equation}
				\tag*{}
			\end{equation}
	(e)\quad En sirkel går gjennom punktene A, B og C.
	Finn senteret og radien til sirkelen.
			\begin{equation}
				\tag*{}
			\end{equation}

\end{document}
