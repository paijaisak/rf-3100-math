\documentclass[12pt, a4paper]{article}
\usepackage[utf8]{inputenc}
\usepackage[leqno]{amsmath}
\usepackage{amsfonts}
\usepackage[a4paper, total={6in, 8in}, margin=0.8in, tmargin=0in]{geometry}
%\usepackage{graphicx}
%\graphicspath{{./images}}

\title{Assignment 02}
\author{Embla Mathea}
\date{\today}

\begin{document}
\maketitle

\begin{center}
	\section*{Exercises}
\end{center}

\subsection*{Exercise 01}
Gitt et plan som går gjennom punktet
$A=\left(-3,\;1,\;\frac{1}{2}\right)$.
Normalvektoren til planet er $\left[2,\;5,\;-1\right]$.\\\\		
	(a)\quad Sett opp den implisitte formelen til planet.
			\begin{gather}
				\tag*{}
					ax+by+cz\;=\;d\\
				\tag*{}
					2x+5y-z\;=\;d\\
				\tag*{}
					2(-3)+5(1)-\frac{1}{2}\;=\;d\\
				\tag*{}
					-6+5-\frac{1}{2}\;=\;d\;=\;-\frac{3}{2}
			\end{gather}
			\begin{equation}
				\tag{Implisitt formel}
					2x+5y-z\;=\;-1.5\\
			\end{equation}
	(b)\quad Hva er avstanden mellom planet og origo?
			\begin{equation}
				\tag*{}
					\frac{d}{\sqrt{a^2+b^2+c^2}}\;=\;
					-\frac{1.5}{\sqrt{2^2+5^2+(-1)^2}}\;=\;
					-\frac{1.5}{\sqrt{30}}\;\approx\;0.274					
			\end{equation}
	(c)\quad Gjør om normalvektoren til en enhetsvektor
	med samme retning, og ta indreproduktet med
	$\vec{A}$. Hva observerer du?
			\begin{gather}
				\tag*{}
					\hat{n}\;=\;\left[
						\frac{2}{\left|n\right|},\;
						\frac{5}{\left|n\right|},\;
						\frac{-1}{\left|n\right|}
					\right]\\
				\tag*{}
						\left|n\right|\;=\;\sqrt{2^2+5^2+(-1)^2}
						\;=\;\sqrt{30}\\
				\tag*{}
					\hat{n}\;=\;\left[
						\frac{2}{\sqrt{30}},\;
						\frac{5}{\sqrt{30}},\;
						\frac{-1}{\sqrt{30}}
					\right]
				\end{gather}
			\begin{equation}
				\tag*{}
					\hat{n}\cdot\vec{a}\;=\;
					\left[
						\frac{2}{\sqrt{30}},\;
						\frac{5}{\sqrt{30}},\;
						\frac{-1}{\sqrt{30}}
					\right]\cdot
					\left[
						-3,\;1,\;\frac{1}{2}
					\right]\;=\;
					\left[
						\frac{-6}{\sqrt{30}}+
						\frac{5}{\sqrt{30}}+
						\frac{-0.5}{\sqrt{30}}
					\right]\;=\;-\frac{1.5}{\sqrt{30}}
			\end{equation}
			
\newpage\quad			
\subsection*{Exercise 03}
Gitt et plan beskrevet med:
	\begin{equation}
		\tag*{}
		2x-y+3z=2
	\end{equation}
Gitt et punkt $A=(3,\;3,\;-1)$ som ligger utenfor planet.\\\\
	(a)\quad Hva er avstanden mellom planet og punktet
			\begin{equation}
				\tag*{}
					|\vec{a}|\;=\;\frac{
					\vec{A}\cdot\vec{n}-d}
					{\left|\vec{n}\right|^2}
			\end{equation}
			\begin{equation}
				\tag*{}
					\left|\vec{a}\right|
					\;=\;\frac{2(3)-3+3(-1)-2}
					{\sqrt{2^2+(-1)^2+3^2}^2}\;=\;
					\frac{6-3-3-2}{14}\;=\;
					-\frac{2}{14}\;=\;-\frac{1}{7}
			\end{equation}
	(b)\quad Finn koordinatene til punktet på planet som
	ligger nærmest $A$.
			\begin{equation}
				\tag*{}
					\vec{P}\;=\;\vec{A}-|
					\vec{a}|\cdot\vec{n}
			\end{equation}
			\begin{gather}
				\tag*{}
					\vec{P}\;=\;\vec{A}-
					\left(-\frac{1}{7}\right)(2,\;-1,\;3)
					\;=\;|3,\;3,\;-1|-\left|
						-\frac{2}{7},\;
						\frac{1}{7},\;
						-\frac{3}{7}
					\right|\\
				\tag*{}
					\;=\;\left|
						\frac{21-(-2)}{7},\;
						\frac{21-1}{7},\;
						\frac{-7-(-3)}{7}
					\right|\;=\;\left|
						\frac{23}{7},\;
						\frac{20}{7},\;
						-\frac{4}{7}
					\right|
			\end{gather}
			\begin{equation}
					\tag*{}
						P\;=\;\left(
							\frac{23}{7},\;
							\frac{20}{7},\;
							-\frac{4}{7}
						\right)
			\end{equation}
\newpage\quad		
\subsection*{Exercise 04}
Gitt fem punkter i rommet:
	\begin{gather}
		\tag*{}
			A\;=\;(1,\;2,\;0)\\
		\tag*{}
			B\;=\;\left(-2,\;-\frac{1}{3},\;2\right)\\
		\tag*{}
			C\;=\;(0,\;2,\;3)\\
		\tag*{}
			D\;=\;\left(3,\;\frac{1}{2},\;0\right)\\
		\tag*{}
			E\;=\;(1,\;1,\;2)
	\end{gather}
	(a)\quad Bruk formlene fra s. 315 i læreboka (eller
	algoritmen fra s. 315-316) til å finne 
	normalvektoren til	planet som passer
	best til disse punktene.
		\begin{flalign*}
			n_x\;=\;&(z_1+z_2)(y_1-y_2)\;+\;
					(z_2+z_3)(y_2-y_3)\;+\;
					(z_3+z_4)(y_3-y_4)\\
					&\quad+\;(z_4+z_5)(y_4-y_5)\;+\;
					(z_5+z_1)(y_5-y_1)\\
			n_y\;=\;&(x_1+x_2)(z_1-z_2)\;+\;
					(x_2+x_3)(z_2-z_3)\;+\;
					(x_3+x_4)(z_3-z_4)\\
					&\quad+\;(x_4+x_5)(z_4-z_5)\;+\;
					(x_5+x_1)(z_5-z_1)\\
			n_z\;=\;&(y_1+y_2)(x_1-x_2)\;+\;
					(y_2+y_3)(x_2-x_3)\;+\;
					(y_3+y_4)(x_3-x_4)\\
					&\quad+\;(y_4+y_5)(x_4-x_5)\;+\;
					(y_5+y_1)(x_5-x_1)
		\end{flalign*}
		\begin{flalign*}
			&\begin{aligned}
				n_x\;=\;
					(0+2)\left(2-\left(-\frac{1}{3}
						\right)\right)\;+\;
					(2+3)\left(-\frac{1}{3}-2\right)\;+\;
					(3+0)\left(2-\frac{1}{2}\right)\\
					+\;	(0+2)\left(\frac{1}{2}-1
					\right)\;+\;(2+0)(1-2)
			\end{aligned}\\
			&\begin{aligned}
				n_y\;=\;
					(1+(-2))(0-2)\;+\;
					(-2+0)(2-3)\;+\;
					(0+3)(3-0)\\+\;
					(3+1)(0-2)\;+\;
					(1+1)(2-0)
			\end{aligned}\\
			&\begin{aligned}
				n_z\;=\;
					\left(2+\left(-\frac{1}{3}\right)\right)
						(1-(-2))\;+\;
					\left(-\frac{1}{3}+2\right)(-2-0)\;+\;
					\left(2+\frac{1}{2}\right)(0-3)\\+\;
					\left(\frac{1}{2}+1\right)(3-1)\;+\;
					(1+2)(1-1)
			\end{aligned}
		\end{flalign*}
		\begin{equation*}
			\begin{aligned}
					n_x\;=\;2\left(\frac{6}{3}+\frac{1}{3}
					\right)+5\left(-\frac{1}{3}-\frac{6}{3}
					\right)+3\left(\frac{4}{2}-\frac{1}{2}
					\right)+2\left(\frac{1}{2}-\frac{2}{2}
					\right)+(2-4)\\
					=\;2\left(\frac{7}{3}\right)+
					5\left(-\frac{7}{3}\right)+
					3\left(\frac{3}{2}\right)+
					2\left(-\frac{1}{2}\right)+(-2)\\								=\;\frac{14}{3}-\frac{35}{3}+
					\frac{9}{2}-1-3
					\;=\;-7+\frac{9}{2}-1\\
					=\;-\frac{-14+9-2}{2}\;=\;-\frac{7}{2}
					\;=\;3.5
			\end{aligned}
		\end{equation*}
		\begin{equation*}
			\begin{aligned}
				n_y\;=\;-1(-2)-2(-1)+3(3)+4(-2)+2(2)\;=\;
				2+2+9-8+4\;=\;9
			\end{aligned}
		\end{equation*}
		\begin{equation*}
			\begin{aligned}
				n_z\;=\;\frac{5}{3}(3)+\frac{5}{3}(-2)
				+\frac{5}{2}(-3)+\frac{3}{2}(2)+3(0)
				\;=\;\frac{15}{3}-\frac{10}{3}
				-\frac{15}{2}+\frac{6}{2}\\=\;
				\frac{5}{3}-\frac{9}{2}\;=\;
				\frac{10-27}{6}\;=\;-\frac{17}{6}
			\end{aligned}
		\end{equation*}
\newpage\quad
		\begin{gather}
			\tag*{}
				\hat{n}\;=\;\left[
					\frac{3.5}{|n|},\;
					\frac{9}{|n|},\;
					\frac{\left(-\frac{17}{6}\right)}{|n|}
				\right]\\
			\tag*{}
				|n|\;=\;\sqrt{
					3.5^2+9^2+\left(-\frac{17}{6}\right)^2
				}\;=\;\sqrt{12.25+81+\frac{289}{36}}
				\;=\;\frac{\sqrt{3646}}{6}
				\;\approx\;10.064\\
			\tag*{}
				\hat{n}\;=\;\left[
						\frac{3.5}{10.064},\;
						\frac{5}{10.064},\;
						\frac{\left(-\frac{17}{6}\right)}
						{10.064}					
				\right]\;=\;\left[
					0.35,\;0.50,\;-0.28
				\right]
		\end{gather}
	(b)\quad Bruk formelene fra s. 316 til å finne $d$.
			\begin{equation*}
				d\;=\;\frac{1}{n}\sum_{i=1}^{n}\left(
					p_i \cdot n
				\right)\;=\;\frac{1}{n}\left(
					\sum_{i=1}^{n}p_i
				\right)\cdot n
			\end{equation*}
			\begin{gather*}
				\sum_{i=1}^{5}p_i\;=\;
					(1,\;2,\;0)\;+\;\left(
						-2,\;-\frac{1}{3},\;2
					\right)\;+\;
					(0,\;2,\;3)\;+\;
					\left(3,\;\frac{1}{2},\;0\right)\;+\;
					(1,\;1,\;2)\;=\;\left
						(3,\;\frac{31}{6},\;7
					\right)\\
				\begin{aligned}
					\frac{1}{5}\left(
						3,\;\frac{31}{6},\;7
					\right)\cdot[0.35,\;0.5,\;-0.28]\;=\;
					\left(
						\frac{3}{5},\;\frac{31}{30},\;
						\frac{7}{5}
					\right)\cdot[0.35,\;0.5,\;-0.28]
					\;\\\approx\;0.21+0.52-0.39\;=\;0.34
				\end{aligned}\\
				d\;=\;0.34
			\end{gather*}
\subsection*{Exercise 05}
Gitt en trekant i 3D-rommet med hjørnene
$A$, $B$ og $C$. Vi får vite at $A$ og $B$ er henholdsvis $(2, 0, 1)$ og $(3, 3, 4)$. Vi får også
vite at vinkelen ved hjørnet $C$ er $50^\circ$. Til slutt får vi vite at lengden fra $B$ til $C$ er 
$4,5$\\\\
	(a)\quad Regn ut lengden $AB$.
			\begin{gather}
				\tag*{}
					\vec{AB}\;=\;[
						3-2,\;3-0,\;4-1
					]\;=\;[1,\;3,\;3]\\
				\tag*{}
					\left|\vec{AB}\right|\;=\;
					\sqrt{1^2+3^2+3^2}\;=\;\sqrt{19}
			\end{gather}
	(b)\quad Bruk sinussetningen for å beregne vinkelen
	ved hjørnet $A$.
			\begin{gather}
				\tag*{}
					\frac{\sin{A}}{l_1}\;=\;
					\frac{\sin{B}}{l_2}\;=\;
					\frac{\sin{C}}{l_3}\\
				\tag*{}
					\frac{\sin{50^\circ}}{\sqrt{19}}\;=\;
					\frac{\sin{A}}{4.5}\\
				\tag*{}
					\sin{A}\;=\;4.5\cdot
					\frac{\sin{50^\circ}}{\sqrt{19}}
					\approx0.79\\
				\tag*{}
					\angle{A}\;=\;\arcsin{0.79}
					\;\approx\;52.19^\circ
			\end{gather}
	(d)\quad Bruk cosinussetningen for å beregne
	lengden $AC$.
			\begin{equation}
				\tag*{}
					b^2\;=\;a^2+c^2-2ac\cdot\cos{B}
			\end{equation}
			\begin{gather}
				\tag*{}
					a=4.5,\quad c=\sqrt{19}\\
				\tag*{}
					\angle{B}=180^\circ-\angle{A}-\angle{C}
					\;=\;77.81^\circ
			\end{gather}
			\begin{gather}
				\tag*{}
					b^2\;=\;4.5^2+\left(\sqrt{19}\right)^2
						-2\cdot4.5\cdot\sqrt{19}
						\cdot\cos{77.81}\\
				\tag*{}
					b^2\;=\;20.25+19-8.28\;=\;30.97					
			\end{gather}
			\begin{equation}
				\tag*{}
					b\;=\;\sqrt{30.97}\;\approx\;5.57
			\end{equation}
\newpage\quad
\subsection*{Exercise 06}
Gitt en trekant i 2D-rommet med hjørnene i A = (2, 5),
	B = (-1, -2) og C = (3, -0,5).\\\\
	(a)\quad Et punkt D har barisentriske koordinater $
	\left(\frac{1}{4}:\frac{2}{5}:\frac{1}{3}\right)
	$ i forhold til trekanten $ABC$. Finn de kartesiske 
	koordinatene til punktet.
		\begin{equation}
			\tag*{}
				D\;=\;b_1v_1+b_2v_2+b_3v_3
		\end{equation}
			\begin{gather}
				\tag*{}
					b_1=\frac{1}{4}\cdot[2,\;5],\qquad
					b_2=\frac{2}{5}\cdot[-1,\;-2],\qquad
					b_3=\frac{1}{3}\cdot[3,\;-0.5]\\
				\tag*{}
					b_1\;=\;\left[
						0.5,\;\frac{5}{4}
					\right],\qquad
					b_2\;=\;\left[
						-\frac{2}{5},\;-\frac{4}{5}
					\right],\qquad
					b_3\;=\;\left[
						1,\;-\frac{1}{6}
					\right]
			\end{gather}
			\begin{equation}
				\tag*{}
					D\;=\;\left(
						0.5-\frac{2}{5}+1,\;
						\frac{5}{4}-\frac{4}{5}-\frac{1}{6}
					\right)\;\approx\;\left(
						1.1,\;0.28
					\right)			
			\end{equation}
	(c)\quad Regn ut de kartesiske koordinatene til
	trekantens tyngdepunkt.
			\begin{equation}
				\tag*{}
					b_1\;=\;\frac{1}{3}\cdot[2,\;5],\qquad
					b_2\;=\;\frac{1}{3}\cdot[-1,\;-2],\qquad
					b_3\;=\;\frac{1}{3}\cdot[3,\;-0.5]
			\end{equation}
			\begin{gather*}
				T\;=\;
					\frac{1}{3}\cdot[2,\;5] +
					\frac{1}{3}\cdot[-1,\;-2] +
					\frac{1}{3}\cdot[3,\;-0.5]\\
				T\;=\;
					\left(
						\frac{2}{3}-\frac{1}{3}+\frac{3}{3}
					,\quad
						\frac{5}{3}-\frac{2}{3}-\frac{1}{6}
					\right)\;\approx\;
					\left(
						1.33,\;0.83
					\right)
			\end{gather*}
	(e)\quad En sirkel går gjennom punktene A, B og C.
	Finn senteret og radien til sirkelen.
			\begin{gather*}
				\vec{BC}\;=\;[3-(-1),\quad-0.5-(-2)]
					\;=\;[4,\;1.5]\\
				\vec{CA}\;=\;[2-3,\quad5-(-0.5)]
					\;=\;[-1,\;5.5]\\
				\vec{AB}\;=\;[-1-2,\quad-2-5]
					\;=\;[-3,\;-7]
			\end{gather*}
%			\begin{gather*}
%				a\;=\;\sqrt{4^2+1.5^2}
%					\;\approx\;4.27\\
%				b\;=\;\sqrt{(-1)^2+5.5^2}
%					\;\approx\;5.59\\
%				c\;=\;\sqrt{(-3)^2+(-7)^2}
%					\;\approx\;7.62\\				
%				a\;<\;c\;>\;b,\quad r\;=\;7.62
%			\end{gather*}
			\begin{gather*}
				\begin{aligned}
					d_1\;=\;-e_2\cdot e_3\;=\;
					-[-1,\;5.5]\cdot[-3,\;-7]\\
					=\;1\cdot3+(-5.5)\cdot-7\\
					=\;3+38.5\;=\;41.5
				\end{aligned}\\
				\begin{aligned}
					d_2\;=\;-e_3\cdot e_1\;=\;
					-[-3,\;-7]\cdot[4,\;1.5]\\
					=\;3\cdot4+7\cdot1.5\\
					=\;12+10.5\;=\;22.5
				\end{aligned}\\
				\begin{aligned}
					d_3\;=\;-e_1\cdot e_2\;=\;
					-[4,\;1.5]\cdot[-1,\;5.5]\\
					=\;-4\cdot-1+-1.5\cdot5.5\\
					=\;4-8.25\;=\;-4.25
				\end{aligned}
			\end{gather*}
			\begin{gather*}
				c_1\;=\;d_2d_3\;=\;22.5\cdot-4.25
					\;=\;-95.625\\
				c_2\;=\;d_3d_1\;=\;-4.25\cdot41.5
					\;=\;-176.375\\
				c_3\;=\;d_1d_2\;=\;41.5\cdot22.5
					\;=\;933.75
			\end{gather*}
			\begin{equation*}
				c\;=\;c_1+c_2+c_3\;=\;661.75
			\end{equation*}
\newpage
			\begin{gather*}
				c_{circ}\;=\;
					\frac{
						(c_2+c_3)v_1+(c_3+c_1)v_2
						+(c_1+c_2)v_3
					}{2c}
					\\
				c_{circ}\;=\;
					\frac{
						757.375\cdot[2,\;5] +
						838.125\cdot[-1,\;-2] +
						-272\cdot[3,\;-0.5]
					}{
						1323.5
					}
					\\
				\begin{aligned}
					c_{circ}\;=\;
						\frac{
							[1514.75,\;3786.875] +
							[-838.125,\;-1676.25] +
							[816,\;136]
						}{
							1323.5					
						}\\=\;
						\frac{
							[1492.625,\;2246.625]
						}{
							1323.5
						}\;\approx\;
						[1.12,\;1.70]
				\end{aligned}
			\end{gather*}
			\begin{gather*}
				\begin{aligned}
					r\;=\;\frac{
						\sqrt{(d_1+d_2)(d_2+d_3)(d_3+d_1)/c}
					}{
						2				
					}\;=\;\frac{
						\sqrt{
							(41.5+22.5)(22.5-4.25)
							(-4.25+41.5)/661.75
						}
					}{
						2
					}\\
					=\;\frac{
						\sqrt{64\cdot18.25\cdot37.25/661.75}
					}{2}\;=\;\frac{
						\sqrt{\frac{43508}{661.75}}
					}{2}\;\approx\;4.05
				\end{aligned}
			\end{gather*}
\end{document}
