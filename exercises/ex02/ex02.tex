\documentclass[12pt, a4paper]{article}
\usepackage[utf8]{inputenc}
\usepackage[leqno]{amsmath}
\usepackage{amsfonts}
\usepackage[a4paper, total={6in, 8in}, margin=0.8in, tmargin=0in]{geometry}

\pagestyle{headings}

\title{Exercises 02}
\author{Embla Mathea}
\date{\today}


\begin{document}
\maketitle

\begin{center}
	\section*{Vectors}	
\end{center}

\subsection*{Exercise 01}

\textit{Let}
\begin{equation}
		\tag*{}
		\mathbf{a} = \left[-3\quad8\right], \qquad			
		\mathbf{b} =
		\begin{bmatrix}
		4\\0\\5
		\end{bmatrix}, \qquad		
		\mathbf{c} =
		\begin{bmatrix}
		16\\-1\\4\\6
		\end{bmatrix}
\end{equation}

\textit{(a) Identify a, b, and c, as row or column vectors, and give the dimension
of each vector}

	\begin{center}
		\textbf{a} is a 2D row vector, \textbf{b} is a 3D column vector 	
		and \textbf{c} is a 4D column vector. \\
	\end{center}

\textit{(b) Compute $b_y + c_w + a_x + b_z$}

	\begin{equation}
		\tag*{}
		0 + 6 + (-3) + 5 = 8
	\end{equation}
	
\subsection*{Exercise 04}

\textit
{
	Identify the following statements as true or false. If the
	statement is false, explain why.\\
}

\textit
{
	(a) The size of a vector in a diagram doesn't matter; we just need
	to draw it in the right place.\\
}

This is false. A vector does not have a position, but the size in a diagram matters.\\

\textit
{
	(b) The displacement expressed by a vector can be visualized as a
	sequence of axially aligned displacements.\\
}

This is true.\\

\textit
{
	(c) These axially aligned displacements from the previous question
	must occur in order.\\
}

This is false. The order of axially aligned displacements must not occur in order. If you add two vectors/displacements, you end up with the same product no matter the order of addition.\\

\textit
{
	 (d) The vector $[x, y]$ gives the displacement from the point $(x,
	 y)$ to the origin.\\
}

This is false. The opposite is true: the vector gives the displacement from the origin to the point.\\

\newpage\quad

\subsection*{Exercise 05}

Evaluate the following vector expressions:\\\\
(a) \quad $-[3 \quad 7]$\\

\begin{equation}
	\tag*{}
	-[3 \quad 7] = [-3 \quad -7]
\end{equation}
(b) \quad $\left|\left|\left[ -12 \quad 5 \right]\right|\right|$\\


\begin{equation}
	\tag*{}
	\sqrt{\left(-12\right)^2 + 5^2} = \sqrt{144 + 25} = 13
\end{equation}
(c) \quad $\left|\left| \left[ 8 \quad -3 \quad \frac{1}{2} \right]
	\right|\right|$


\begin{equation}
	\tag*{}
	\left|\left| \left[ 8 \quad -3 \quad \frac{1}{2} \right]
	\right|\right| \; = \; \sqrt{64 + 9 + \frac{1}{4}} \; = \; \sqrt{73.5}
	\approx 8.56
\end{equation}

\subsection*{Exercise 06}
Normalize the following vectors:\\\\
(a) \quad $\left[12 \quad 5 \right]$\\

\begin{equation}
	\tag*{}	
	\frac
	{
		\left[12 \quad 5\right]
	}
	{
		\left|\left|\left[12 \quad 5\right]\right|\right|
	} \; = \;	
	\frac
	{
		\left[12 \quad 5\right]
	}
	{
		\sqrt{12^2 + 5^2}	
	} \; = \;
	\frac
	{
		\left[12 \quad 5\right]
	}
	{
		\sqrt{144 + 25}
	} \; = \;
	\frac
	{
		\left[12 \quad 5\right]
	}
	{
		13
	} \; = \;
	\left[
		\frac
		{
			12			
		}
		{
			13
		}
		\quad
		\frac
		{
			5		
		}
		{
			13
		}
	\right] \; \approx \;
	\left[0.923 \quad 0.385 \right]
\end{equation}\\
(b) \quad $\left[0 \quad 743.632 \right]$\\
\begin{equation}
	\tag*{}
	\frac
	{
		\left[0 \quad 743.632\right]
	}
	{
		\left|\left|\left[0 \quad 743.632\right]\right|\right|
	} \; = \;
	\frac
	{
		\left[0 \quad 743.632\right]
	}
	{
		\sqrt{0^2 + 743.632^2}
	} \; = \;
	\frac
	{
		\left[0 \quad 743.632\right]
	}
	{
		743.632
	} \; = \;
	\left[
		0 \quad 1	
	\right]
\end{equation}\\
(c) \quad $\left[8 \quad -3 \quad \frac{1}{2} \right]$\\
\begin{equation}
	\tag*{}
	\frac
	{
		\left[8 \quad -3 \quad \frac{1}{2}\right]
	}
	{
		\left|\left|\left[
		8 \quad -3 \quad \frac{1}{2}
		\right]\right|\right|
	} \; = \;
	\frac
	{
		\left[8 \quad -3 \quad \frac{1}{2}\right]
	}
	{
		\sqrt{8^2 + (-3)^2 + \frac{1}{2}^2}
	} \; = \;
	\frac
	{
		\left[8 \quad -3 \quad \frac{1}{2}\right]
	}
	{
		\sqrt{73.25} \approx 8.56
	}
\end{equation}
\begin{equation}
	\tag*{}
	\; = \;	
	\left[
		\frac{8}{\sqrt{73.25}}\quad
		\frac{-3}{\sqrt{73.25}}\quad
		\frac{0.5}{\sqrt{73.25}}
	\right]
	\approx\\
	\left[
		0.935 \quad -0.351 \quad 0.058
	\right]
\end{equation}
\newpage\quad

\subsection*{Exercise 07}
Evaluate the following vector expressions:\\\\
	(a)\quad $\left[7 \quad -2 \quad -3\right]$
		$+\left[6 \quad 6 \quad -4\right]$
		\begin{equation}
			\tag*{}
			\left[
				7+6 \quad -2+6 \quad -3-4
			\right] \; = \;
			\left[
				13 \quad 4 \quad -7
			\right]
		\end{equation}
	(b)\quad $\left[2 \quad 9 \quad -1\right]$
		$+\left[-2 \quad -9 \quad 1\right]$
	\begin{equation}
		\tag*{}
		\left[2-2 \quad 9-9 \quad 1-1\right]
		= \left[0 \quad 0 \quad 0\right]
	\end{equation}

\subsection*{Exercise 08}
Compute the distance between the following pairs of points:

	\begin{flalign}
		\tag{a}
		&
		\begin{bmatrix}
				10\\
				6
		\end{bmatrix} ,\;
		\begin{bmatrix}
			-14\\
			30
		\end{bmatrix}
		&&
	\end{flalign}
	
	\begin{equation}
		\tag*{}
		\left\Vert
			\begin{bmatrix}
				-14 - 10 \\
				30 - 6
			\end{bmatrix}
		\right\Vert \; = \;
		\left\Vert
			\begin{bmatrix}
				-24 \\
				24
			\end{bmatrix}
		\right\Vert \; = \;
		\sqrt{(24)^2 + 24^2} \; \approx \; 33.94
	\end{equation}

	\begin{flalign}
		\tag{b}
		&
		\begin{bmatrix}
			0\\
			0
		\end{bmatrix} ,\;
		\begin{bmatrix}
			-12\\
			5
		\end{bmatrix}
		&&
	\end{flalign}
	
	\begin{equation}
		\tag*{}
		\left|\Vert\begin{bmatrix}
			-12\\
			5
		\end{bmatrix}\right\Vert \; = \;
		\sqrt{(-12)^2 + 5^2} \; = \;
		\sqrt{144 + 25} \; = \; 13
	\end{equation}
	
	\begin{flalign}
		\tag{c}
		&
		\begin{bmatrix}
			3\\
			10\\
			7
		\end{bmatrix} ,\;
		\begin{bmatrix}
			8\\
			-7\\
			4
		\end{bmatrix}
		&&
	\end{flalign}
	
	\begin{equation}
		\tag*{}
		\left\Vert\begin{bmatrix}
			8 - 3 \\
			-7 - 10 \\
			4 - 7
		\end{bmatrix}\right\Vert \; = \;
		\left\Vert\begin{bmatrix}
			5 \\
			-17 \\
			-3
		\end{bmatrix}\right\Vert \; = \;
		\sqrt{5^2 + (-17)^2 + (-3)^2} \; = \;
		\sqrt{25 + 289 + 9} \; \approx \; 17.97
	\end{equation}
	
\subsection*{Exercise 09}
Evaluate the following vector expressions:

	\begin{flalign}
		\tag{a}
		&
			\begin{bmatrix}
				2\\
				6			
			\end{bmatrix} \;\cdot\;
			\begin{bmatrix}
				-3\\
				8			
			\end{bmatrix}							
		&&
	\end{flalign}
	
		\begin{equation}
			\tag*{}
			(2)(-3) + (6)(8) \,=\, -6+48 \;=\; 42
		\end{equation}
	
	\begin{flalign}
		\tag{b}
		&
			-7\left[1 \quad 2\right]\;\cdot\;\left[11 \quad -4\right]
		&&
	\end{flalign}
	
		\begin{align}
			\tag*{}
			& \left[-7 \quad -14\right] \cdot \left[11 \quad -4\right]
				\;=\; (-7)(11) + (-14)(-4)\\
			\tag*{}
			& \qquad\qquad\qquad\qquad\qquad\qquad=\; -77 + 56 \;=\; -21
		\end{align}

\newpage\quad

\subsubsection*{Exercise 13}
Calculate $\mathbf{a \times b}$ and $\mathbf{b \times a}$ for the following vectors:

	\begin{flalign}
		\tag{a}
		&		
			\mathbf{a} =
			\begin{bmatrix}
				0 \\ -1 \\ 0
			\end{bmatrix},\;
			\mathbf{b} =
			\begin{bmatrix}
				0 \\ 0 \\ 1
			\end{bmatrix}
		&&
	\end{flalign}
		
		\begin{equation}
			\tag*{}
			\begin{bmatrix}
				-1\cdot1 \;-\; 0\cdot0 \\
				 0\cdot0 \;-\; 0\cdot1 \\
				 0\cdot0 \;-\; -1\cdot0
			\end{bmatrix} \; = \; 0
		\end{equation}
	
	\begin{flalign}
		\tag{b}
		&
			\mathbf{a} =
			\begin{bmatrix}
				-2 \\ 4 \\ 1
			\end{bmatrix}	,\;
			\mathbf{b} =
			\begin{bmatrix}
				1 \\ -2 \\ -1
			\end{bmatrix}
		&&
	\end{flalign}
	
		\begin{equation}
			\tag*{}
			\begin{bmatrix}
				(4)(-1) - (1)(-2)\\
				(1)(1) - (-2)(-1)\\
				(-2)(-2) - (4)(1)			
			\end{bmatrix} \;=\;
			\begin{bmatrix}
				-4 + 2 \\ 1 - 2 \\ 4 - 4			
			\end{bmatrix} \;=\;
			\begin{bmatrix}
				-2 \\ -1 \\ 0
			\end{bmatrix}
		\end{equation}
	
	\begin{flalign}
		\tag{c}
		&
			\mathbf{a} =
			\begin{bmatrix}
				3 \\ 10 \\ 7
			\end{bmatrix},\;
			\mathbf{b}
			\begin{bmatrix}
				8 \\ -7 \\ 4			
			\end{bmatrix}				
		&&
	\end{flalign}
	
	\begin{equation}
		\tag*{}
		\begin{bmatrix}
			(10)(4) - (7)(-7)\\
			(7)(8) - (3)(4)\\
			(3)(-7) - (10)(8)
		\end{bmatrix}\;=\;
		\begin{bmatrix}
			40 + 49 \\ 56 - 12 \\ -21 -80
		\end{bmatrix}\;=\;
		\begin{bmatrix}
			89 \\ 44 \\ -101
		\end{bmatrix}
	\end{equation}

\end{document}
