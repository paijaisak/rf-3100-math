\documentclass[12pt, a4paper]{article}
\usepackage[utf8]{inputenc}
\usepackage[leqno]{amsmath}
\usepackage{amsfonts}
\usepackage[a4paper, total={6in, 8in}, margin=0.8in, tmargin=0in]{geometry}

\pagestyle{headings}

\title{Exercises 02}
\author{Embla Mathea}
\date{\today}


\begin{document}
\maketitle

\begin{center}
	\section*{Vectors}	
\end{center}

\subsection*{Exercise 01}

\quad

\textit{Let}
\begin{equation}
		\tag*{}
		\mathbf{a} = \left[-3\quad8\right], \qquad			
		\mathbf{b} =
		\begin{bmatrix}
		4\\0\\5
		\end{bmatrix}, \qquad		
		\mathbf{c} =
		\begin{bmatrix}
		16\\-1\\4\\6
		\end{bmatrix}
\end{equation}

\textit{(a) Identify a, b, and c, as row or column vectors, and give the dimension
of each vector}

	\begin{center}
		\textbf{a} is a 2D row vector, \textbf{b} is a 3D column vector 	
		and \textbf{c} is a 4D column vector. \\
	\end{center}

\textit{(b) Compute $b_y + c_w + a_x + b_z$}

	\begin{equation}
		\tag*{}
		0 + 6 + (-3) + 5 = 8
	\end{equation}
	
\subsection*{Exercise 04}

\quad

\textit
{
	Identify the following statements as true or false. If the
	statement is false, explain why.\\
}

\textit
{
	(a) The size of a vector in a diagram doesn't matter; we just need
	to draw it in the right place.\\
}

This is false. A vector does not have a position, but the size in a diagram matters.\\

\textit
{
	(b) The displacement expressed by a vector can be visualized as a
	sequence of axially aligned displacements.\\
}

This is true.\\

\textit
{
	(c) These axially aligned displacements from the previous question
	must occur in order.\\
}

This is false. The order of axially aligned displacements must not occur in order. If you add two vectors/displacements, you end up with the same product no matter the order of addition.\\

\textit
{
	 (d) The vector $[x, y]$ gives the displacement from the point $(x,
	 y)$ to the origin.\\
}

This is false. The opposite is true: the vector gives the displacement from the origin to the point.\\

\end{document}
