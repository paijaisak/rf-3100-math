\documentclass[12pt, a4paper]{article}
\usepackage[utf8]{inputenc}
\usepackage[leqno]{amsmath}
\usepackage{amsfonts}
\usepackage[a4paper, total={6in, 8in}, margin=0.8in, tmargin=0in]{geometry}
\usepackage{graphicx}
\graphicspath{{./images}}
%\usepackage{parskip}
%\setlength{\parindent}{0pt}

\title{Exercises 07}
\author{Embla Mathea}
\date{\today}

\begin{document}
\maketitle

\begin{center}
	\section*{Polar Coordinate Systems}	
\end{center}

\subsection*{Exercise 01}

Plot and label the points with the following polar coordinates:

\begin{center}
	\begin{tabular}{ c c c c c }
		 (a) $(2, 60^\circ)$ & (b) $(5, 195^\circ)$ &
		 (c) $(3, -45^\circ)$ & (d) $(-2.75, 300^\circ)$\\
		 (e) $\left(4, \frac{\pi}{6}rad\right)$ &
		 (f) $\left(1, 4\frac{\pi}{3}rad\right)$ &
		 (g) $\left(\frac{-5}{2}, \frac{\pi}{2}rad\right)$
	\end{tabular}
\end{center}

Cannot solve.

\subsection*{Exercise 02}

Convert the following 2D polar coordinates to canonical form:\\

	(a)\quad $(4,\;207^\circ)$
		\begin{equation}
			\tag*{}
			(4,\;207^\circ-360^\circ)\;=\;(4,\;-153^\circ)
		\end{equation}		
	
	(b)\quad $(-5,\;-720^\circ)$
		\begin{gather}
			\tag*{}
			(-5,\;-720^\circ)\;=\;(5,\;-720^\circ+180^\circ)
			\;=\;(5,\;-540^\circ+360^\circ)\\
			\tag*{}
			\;=\;(5,\;-180+360^\circ)\;=\;(5,\;180^\circ)
		\end{gather}
	
	(c)\quad $(0,\;45.2^\circ)$
		\begin{equation}
			\tag*{}
			(0,\;45.2^\circ)\;=\;(0,\;0^\circ)
		\end{equation}
		
	(d)\quad $\left(12.6,\;\frac{11\pi}{4}rad\right)$
		\begin{equation}
			\tag*{}
			-\pi<\theta\geq\pi
		\end{equation}
		\begin{equation}
			\tag*{}
			\frac{11\pi}{4}>\pi,\quad
			\frac{11\pi}{4}-2\pi\;=\;
			\frac{11\pi}{4}-\frac{8\pi}{4}\;=\;
			\frac{3\pi}{4}
		\end{equation}
		\begin{equation}
			\tag*{}
			\left(12.6,\;\frac{11\pi}{4}rad\right)\;=\;
			\left(12.6,\;\frac{3\pi}{4}rad\right)
		\end{equation}
	\newpage\quad
\subsection*{Exercise 03}
Convert the following 2D polar coordinates to Cartesian form:\\

	(a)\quad $(1,\;45^\circ)$
			\begin{equation}
				\tag*{}
				\mathbf{x}\;=\;r\cos\theta,\quad
				\mathbf{y}\;=\;r\sin\theta
			\end{equation}
			\begin{gather}
				\tag*{}
				x\;=\;\cos45^\circ \;\approx\; 0.71 
				\\
				\tag*{}
				y\;=\;\sin45^\circ	\;\approx\; 0.71		
			\end{gather}
	
	(b)\quad $(3,\;0^\circ)$\\\\
	Since the radius is 3 and the 0 degree-angle $\theta$ is on
	the x-axis, the Cartesian coordinates are $(0, 3)$.\\
	
	(c)\quad $(4,\;90^\circ)$\\\\
	A 90-degree angle $\theta$ aligns with the y-axis, the
	Cartesian coordinates are $(0, 4)$.\\
	
	(d)\quad $(10,\;-30^\circ)$
		\begin{gather}
			\tag*{}
			x\;=\;r\cos-30^\circ \;\approx\; 8.66\\
			\tag*{}
			y\;=\;r\sin-30^\circ \;=\; -5.00
		\end{gather}
	
	(e)\quad $(5.5,\;\pi\;rad)$\\\\
		Because\nolinebreak
		\begin{equation}
			\tag*{}
			\pi\;rad\;=\;180^\circ,
		\end{equation}
		\quad the point must be on the negative x-axis.
		Therefore, the Cartesian coordinates are $(-5.5,\;0)$
		
\subsection*{Exercise 04}
\newpage\quad
\subsubsection*{Exercise 05}

Convert the following 2D Cartesian coordinates to
(canonical) polar form:\\
	
	(a)\quad$(10,\;20)$
		\begin{figure}[h]
			\centering
			\includegraphics[scale=0.5]{atan2}
		\end{figure}
		
		\begin{gather}
			\tag*{}
				r\;=\;\sqrt{10^2 + 20^2} \\
			\tag*{}
				\theta\;=\;atan2(y,\;x)=atan2(20,\;10)
		\end{gather}
		\begin{gather}
			\tag*{}
				r\;=\;\sqrt{500}\;\approx\;22.36\\
			\tag*{}
				\arctan\left(\frac{y}{x}\right)\;=\;
				\arctan(2)\;\approx\;63.43^\circ
		\end{gather}
		\begin{equation}
			\tag*{}
				(10,\;20)\;=\;(22.36,\;63.43^\circ)
		\end{equation}\\
	(b)\quad$(-12,\;-5)$\\
	(c)\quad$(0,\;4.5)$\\
	(d)\quad$(-3,\;4)$\\
		\begin{equation}
			\tag*{}
				r\;=\;\sqrt{(-3)^2+4^2}\;=\;5
		\end{equation}
		\begin{gather}
			\tag*{}
				\arctan\left(\frac{y}{x}\right)+180^\circ\;=\;
				\arctan\left(-\frac{4}{3}\right)+180^\circ\\
			\tag*{}
				\;\approx\;-53.13^\circ+180^\circ
				\;\approx126.87^\circ
		\end{gather}
		\begin{equation}
			\tag*{}
				(-3,\;4)\;=\;(5,\;126.87^\circ)
		\end{equation}\\
	(e)\quad$(0,\;0)$++
	(f)\quad$(-5280,\;0)$++
	
\newpage\quad
\subsection*{Exercise 06}
Convert the following cylindrical coordinates to
Cartesian form:\\
	(a)\quad$(4,\;120^\circ,\;5)$
		\begin{gather}
			\tag*{}
				(r,\;\theta,\;z)\\
			\tag*{}
				x\;=\;r\cos\theta\qquad y\;=\;r\sin\theta
		\end{gather}
		\begin{gather}
			\tag*{}
				x\;=\;4\cdot\cos(120^\circ)\qquad
				y\;=\;4\cdot\sin(120^\circ)\\
			\tag*{}
				x\;=\;4\cdot-\frac{1}{2}\qquad
				y\;=\;4\cdot0.866\\
			\tag*{}
				x\;=\;-2 \qquad y\;\approx\;3.46
		\end{gather}
		\begin{equation}
			\tag*{}
				(4,\;120^\circ,\;5)\;=\;(-2,\;3.46,\;5)
		\end{equation}
			
	(b)\quad$(2,\;45^\circ,\;-1)$\\
	(c)\quad$\left(6,\;-\frac{\pi}{6},\;-3\right)$\\
	(d)\quad$(3,\;3\pi,\;1)$
	
\subsection*{Exercise 07}
Convert the following 3D Cartesian coordinates to
(canonical) cylindrical form:\\
	(a)\quad$(1,\;1,\;1)$
		\begin{gather}
			\tag*{}
				r\;=\;\sqrt{2}\\
			\tag*{}
				\theta\;=\;\arctan(1)=45^\circ\\
			\tag*{}
				z\;=\;1
		\end{gather}
		\begin{equation}
			\tag*{}
				(1,\;1,\;1)\;=\;(\sqrt{2},\;45^\circ,\;1)
		\end{equation}
	(b)\quad$(0,\;-5,\;2)$\\
	(c)\quad$(-3,\;4,\;-7)$\\
		\begin{gather}
			\tag*{}
				r\;=\;\sqrt{(-3)^2+4^2}\;=\;5\\
			\tag*{}
				\theta\;=\; \arctan2\left(\frac{y}{x}\right)
				\;=\;\arctan(\frac{4}{-3})+180^\circ\;\approx\;
				-53.13^\circ+180^\circ\;=\;126.87^\circ\\
			\tag*{}
				z\;=\;-7
		\end{gather}
		\begin{equation}
			\tag*{}
				(-3,\;4,\;-7)\;=\;(5,\;126.87^\circ,\;-7)
		\end{equation}
	(d)\quad$(0,\;0,\;-3)$

\newpage\quad
\subsection*{Exercise 08}
Convert the following spherical coordinates
$(r,\;\theta,\;\phi)$ to Cartesian form according to the standard mathematical convention:\\\\
	(a)\quad$\left(4,\;\frac{\pi}{3},\;\frac{3\pi}{4}\right)$
		\begin{equation}
			\tag*{}
				x\;=\;r\;\sin\phi\;\cos\theta\qquad
				y\;=\;r\;\sin\phi\;\sin\theta\qquad
				z\;=\;r\cdot\cos\phi
		\end{equation}
		\begin{gather}
			\tag*{}
				x\;=\;4\cdot\sin\left(\frac{3\pi}{4}\right)\cdot
					\cos\left(\frac{\pi}{3}\right)
					\;\approx1.414\\
			\tag*{}
				y\;=\;4\cdot\sin\left(\frac{3\pi}{4}\right)\cdot
					\sin\left(\frac{\pi}{3}\right)
					\;\approx\;2.449\\
			\tag*{}
				z\;=\;4\cdot\cos\left(\frac{3\pi}{4}\right)
				\;\approx\;-2.828
		\end{gather}
		\begin{equation}
			\tag*{}
				\left(4,\;\frac{\pi}{3},\;\frac{3\pi}{4}\right)
				\;=\;(1.414,\;2.449,\;-2.828)
		\end{equation}
	(b)\quad$\left(5,\;-\frac{5\pi}{6},\;\frac{\pi}{3}\right)$\\
	(c)\quad$\left(2,\;-\frac{\pi}{6},\;\pi\right)$\\
	(d)\quad$\left(8,\;\frac{9\pi}{4},\;\frac{\pi}{6}\right)$
	
\subsection*{Exercise 09}

\subsection*{Exercise 10}
Convert the following 3D Cartesian coordinates to (canonical) spherical form using our modified convention:\\\\
	(a)\quad$\left(\sqrt{2},\;2\sqrt{3},\;-\sqrt{2}\right)$
		\begin{equation}
			\tag*{}
				r\;=\;\sqrt
				{
					\left(\sqrt{2}\right)^2+
					\left(2\sqrt{3}\right)^2+
					\left(-\sqrt{2}\right)^2
				}, \qquad
				h\;=\;atan2(x,\;z), \qquad
				p\;=\;\arcsin\left(\frac{-y}{r}\right)
		\end{equation}
		\begin{gather}
			\tag*{}
				r\;=\;\sqrt{2+12+2}\;=\;4\\
			\tag*{}
				h\;=\;\arctan\left(\frac
				{
				\sqrt{2}}{-\sqrt{2}
				}
				=-1\right)+180^\circ
				\;=\;-45^\circ+180^\circ\;=\;135^\circ\\
			\tag*{}
				p\;=\;\arcsin\left(
					\frac{-y}{r}
				\right)\;=\;\arcsin\left(\frac{
					-\left(
					2\sqrt{3}\right)
				}{\sqrt{2}}\right)\;=\;\arcsin\left(-\frac{
					\sqrt{3}}{2
				}\right)\;=\;-60^\circ
		\end{gather}
		\begin{equation}
			\tag*{}
				\left(\sqrt{2},\;2\sqrt{3},\;-\sqrt{2}\right)
				\;=\;(4,\;135^\circ,\;-60^\circ)
		\end{equation}
\end{document}
