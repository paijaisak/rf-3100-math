\documentclass[12pt, a4paper]{article}
\usepackage[utf8]{inputenc}
\usepackage[leqno]{amsmath}
\usepackage{amsfonts}
\usepackage[a4paper, total={6in, 8in}, margin=0.8in, tmargin=0in]{geometry}
%\usepackage{parskip}
%\setlength{\parindent}{0pt}

\title{Exercises 07}
\author{Embla Mathea}
\date{\today}

\begin{document}
\maketitle

\begin{center}
	\section*{Polar Coordinate Systems}	
\end{center}

\subsection*{Exercise 01}

Plot and label the points with the following polar coordinates:

\begin{center}
	\begin{tabular}{ c c c c c }
		 (a) $(2, 60^\circ)$ & (b) $(5, 195^\circ)$ &
		 (c) $(3, -45^\circ)$ & (d) $(-2.75, 300^\circ)$\\
		 (e) $\left(4, \frac{\pi}{6}rad\right)$ &
		 (f) $\left(1, 4\frac{\pi}{3}rad\right)$ &
		 (g) $\left(\frac{-5}{2}, \frac{\pi}{2}rad\right)$
	\end{tabular}
\end{center}

Cannot solve.

\subsection*{Exercise 02}

Convert the following 2D polar coordinates to canonical form:\\

	(a)\quad $(4,\;207^\circ)$
		\begin{equation}
			\tag*{}
			(4,\;207^\circ-360^\circ)\;=\;(4,\;-153^\circ)
		\end{equation}		
	
	(b)\quad $(-5,\;-720^\circ)$
		\begin{gather}
			\tag*{}
			(-5,\;-720^\circ)\;=\;(5,\;-720^\circ+180^\circ)
			\;=\;(5,\;-540^\circ+360^\circ)\\
			\tag*{}
			\;=\;(5,\;-180+360^\circ)\;=\;(5,\;180^\circ)
		\end{gather}
	
	(c)\quad $(0,\;45.2^\circ)$
		\begin{equation}
			\tag*{}
			(0,\;45.2^\circ)\;=\;(0,\;0^\circ)
		\end{equation}
		
	(d)\quad $\left(12.6,\;\frac{11\pi}{4}rad\right)$
		\begin{equation}
			\tag*{}
			-\pi<\theta\geq\pi
		\end{equation}
		\begin{equation}
			\tag*{}
			\frac{11\pi}{4}>\pi,\quad
			\frac{11\pi}{4}-2\pi\;=\;
			\frac{11\pi}{4}-\frac{8\pi}{4}\;=\;
			\frac{3\pi}{4}
		\end{equation}
		\begin{equation}
			\tag*{}
			\left(12.6,\;\frac{11\pi}{4}rad\right)\;=\;
			\left(12.6,\;\frac{3\pi}{4}rad\right)
		\end{equation}
	\newpage\quad
\subsection*{Exercise 03}
Convert the following 2D polar coordinates to Cartesian form:\\

	(a)\quad $(1,\;45^\circ)$
			\begin{equation}
				\tag*{}
				\mathbf{x}\;=\;r\cos\theta,\quad
				\mathbf{y}\;=\;r\sin\theta
			\end{equation}
			\begin{gather}
				\tag*{}
				x\;=\;\cos45^\circ \;\approx\; 0.71 
				\\
				\tag*{}
				y\;=\;\sin45^\circ	\;\approx\; 0.71		
			\end{gather}
	
	(b)\quad $(3,\;0^\circ)$\\\\
	Since the radius is 3 and the 0 degree-angle $\theta$ is on
	the x-axis, the Cartesian coordinates are $(0, 3)$.\\
	
	(c)\quad $(4,\;90^\circ)$\\\\
	A 90-degree angle $\theta$ aligns with the y-axis, the
	Cartesian coordinates are $(0, 4)$.\\
	
	(d)\quad $(10,\;-30^\circ)$
		\begin{gather}
			\tag*{}
			x\;=\;r\cos-30^\circ \;\approx\; 8.66\\
			\tag*{}
			y\;=\;r\sin-30^\circ \;=\; -5.00
		\end{gather}
	
	(e)\quad $(5.5,\;\pi\;rad)$\\\\
		Because\nolinebreak
		\begin{equation}
			\tag*{}
			\pi\;rad\;=\;180^\circ,
		\end{equation}
		\quad the point must be on the negative x-axis.
		Therefore, the Cartesian coordinates are $(-5.5,\;0)$
		
\end{document}
